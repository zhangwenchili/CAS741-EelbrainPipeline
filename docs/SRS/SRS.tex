% THIS DOCUMENT IS TAILORED TO REQUIREMENTS FOR SCIENTIFIC COMPUTING.  IT SHOULDN'T
% BE USED FOR NON-SCIENTIFIC COMPUTING PROJECTS
\documentclass[12pt]{article}

\usepackage{amsmath, mathtools}
\usepackage{amsfonts}
\usepackage{amssymb}
\usepackage{graphicx}
\usepackage{colortbl}
\usepackage{xr}
\usepackage{hyperref}
\usepackage{longtable}
\usepackage{xfrac}
\usepackage{tabularx}
\usepackage{float}
\usepackage{siunitx}
\usepackage{booktabs}
\usepackage{caption}
\usepackage{pdflscape}
\usepackage{afterpage}

\usepackage[round]{natbib}

%\usepackage{refcheck}

\hypersetup{
    bookmarks=true,         % show bookmarks bar?
      colorlinks=true,       % false: boxed links; true: colored links
    linkcolor=red,          % color of internal links (change box color with linkbordercolor)
    citecolor=green,        % color of links to bibliography
    filecolor=magenta,      % color of file links
    urlcolor=cyan           % color of external links
}

\input{../Comments.text}
\input{../Common.text}

% For easy change of table widths
\newcommand{\colZwidth}{1.0\textwidth}
\newcommand{\colAwidth}{0.13\textwidth}
\newcommand{\colBwidth}{0.82\textwidth}
\newcommand{\colCwidth}{0.1\textwidth}
\newcommand{\colDwidth}{0.05\textwidth}
\newcommand{\colEwidth}{0.8\textwidth}
\newcommand{\colFwidth}{0.17\textwidth}
\newcommand{\colGwidth}{0.5\textwidth}
\newcommand{\colHwidth}{0.28\textwidth}

% Used so that cross-references have a meaningful prefix
\newcounter{defnum} %Definition Number
\newcommand{\dthedefnum}{GD\thedefnum}
\newcommand{\dref}[1]{GD\ref{#1}}
\newcounter{datadefnum} %Datadefinition Number
\newcommand{\ddthedatadefnum}{DD\thedatadefnum}
\newcommand{\ddref}[1]{DD\ref{#1}}
\newcounter{theorynum} %Theory Number
\newcommand{\tthetheorynum}{TM\thetheorynum}
\newcommand{\tref}[1]{TM\ref{#1}}
\newcounter{tablenum} %Table Number
\newcommand{\tbthetablenum}{TB\thetablenum}
\newcommand{\tbref}[1]{TB\ref{#1}}
\newcounter{assumpnum} %Assumption Number
\newcommand{\atheassumpnum}{A\theassumpnum}
\newcommand{\aref}[1]{A\ref{#1}}
\newcounter{goalnum} %Goal Number
\newcommand{\gthegoalnum}{GS\thegoalnum}
\newcommand{\gsref}[1]{GS\ref{#1}}
\newcounter{instnum} %Instance Number
\newcommand{\itheinstnum}{IM\theinstnum}
\newcommand{\iref}[1]{IM\ref{#1}}
\newcounter{reqnum} %Requirement Number
\newcommand{\rthereqnum}{R\thereqnum}
\newcommand{\rref}[1]{R\ref{#1}}
\newcounter{nfrnum} %NFR Number
\newcommand{\rthenfrnum}{NFR\thenfrnum}
\newcommand{\nfrref}[1]{NFR\ref{#1}}
\newcounter{lcnum} %Likely change number
\newcommand{\lthelcnum}{LC\thelcnum}
\newcommand{\lcref}[1]{LC\ref{#1}}

\usepackage{fullpage}

\newcommand{\deftheory}[9][Not Applicable]
{
\newpage
\noindent \rule{\textwidth}{0.5mm}

\paragraph{RefName: } \textbf{#2} \phantomsection 
\label{#2}

\paragraph{Label:} #3

\noindent \rule{\textwidth}{0.5mm}

\paragraph{Equation:}

#4

\paragraph{Description:}

#5

\paragraph{Notes:}

#6

\paragraph{Source:}

#7

\paragraph{Ref.\ By:}

#8

\paragraph{Preconditions for \hyperref[#2]{#2}:}
\label{#2_precond}

#9

\paragraph{Derivation for \hyperref[#2]{#2}:}
\label{#2_deriv}

#1

\noindent \rule{\textwidth}{0.5mm}

}

\begin{document}

\title{Software Requirements Specification \\ for \progname{}} 
\author{\authname}
\date{}
	
\maketitle

~\newpage

\pagenumbering{roman}

\tableofcontents


~\newpage








\section*{Revision History}


\begin{tabularx}{\textwidth}{p{3cm}p{2cm}X}
\toprule {\bf Date} & {\bf Version} & {\bf Notes}\\
\midrule
    Jan 28, 2026 & 1.0 & Initial draft \\
\bottomrule
\end{tabularx}


~\newpage








\section{Reference Material}


This section records information for easy reference.




\subsection{Table of Units}


Throughout this document SI (Syst\`{e}me International d'Unit\'{e}s) is employed as the unit system.  In addition to the basic units, several derived units are used as described below.  For each unit, the symbol is given followed by a description of the unit and the SI name.


~\newline


\renewcommand{\arraystretch}{1.2}
\begin{table}[ht]
\noindent \begin{tabular}{l l l} 
\toprule		
\textbf{symbol} & \textbf{quantity} & \textbf{SI name}\\
\midrule 
    \si{\hertz} & frequency & hertz\\
    \si{\micro\volt} & potential & microvolt\\
\bottomrule
\end{tabular}
\caption{SI units used in this document}
\end{table}




\subsection{Table of Symbols}


The table that follows summarizes the symbols used in this document along with their units. The symbols are listed in alphabetical order.


\renewcommand{\arraystretch}{1.2}
%\noindent \begin{tabularx}{1.0\textwidth}{l l X}
\noindent \begin{longtable*}{l l p{12cm}} \toprule
	extbf{symbol} & \textbf{unit} & \textbf{description}\\
\midrule 
    $\mathbf{A}$ & -- & Forward model / linear operator mapping sources to sensors.\\
    $\tilde{\mathbf{A}}$ & -- & Whitened forward operator.\\
    $f_s$ & \si{\hertz} & Sampling frequency of the M/EEG recordings.\\
    $\mathbf{L}$ & -- & Forward model operator used in the linear source estimation model.\\
    $N_c$ & -- & Number of M/EEG sensors used to record the measurements.\\
    $\mathcal{R}(\mathbf{x})$ & -- & Regularization term in the optimization formulation.\\
    $\mathbf{W}$ & -- & Inverse operator mapping sensor measurements to source estimates.\\
    $\hat{\mathbf{x}}$ & -- & Estimated source vector in the regularized linear inverse problem.\\
    $\mathbf{y}$ & \si{\micro\volt} (EEG) & Observed data vector used in the linear inverse formulation.\\
    $\mathbf{y}_a(t)$ & \si{\micro\volt} (EEG) & Aggregated sensor-level representation derived from $\mathbf{y}(t)$.\\
\bottomrule
\end{longtable*}




\subsection{Abbreviations and Acronyms}


\renewcommand{\arraystretch}{1.2}
\begin{tabular}{l l} 
    \toprule		
    \textbf{symbol} & \textbf{description}\\
    \midrule 
        A & Assumption\\
        DD & Data Definition\\
        GD & General Definition\\
        GS & Goal Statement\\
        IM & Instance Model\\
        LC & Likely Change\\
        PS & Physical System Description\\
        R & Requirement\\
        SRS & Software Requirements Specification\\
        % \progname{} & Eelbrain pipeline\\
        TM & Theoretical Model\\
        BIDS & Brain Imaging Data Structure\\
        MEG & Magnetoencephalography\\
        EEG & Electroencephalography\\
        MRI & Magnetic Resonance Imaging\\
    \bottomrule
\end{tabular}\\




\subsection{Mathematical Notation}


\newpage








\section{Introduction}


In neuroscience experiments, participants are typically presented with controlled stimuli --- such as images, sounds, or words --- while their brain activity is recorded. These experiments aim to uncover how the brain responds to sensory or cognitive events. Magnetoencephalography (MEG) and electroencephalography (EEG) are commonly used techniques that measure the magnetic fields and electrical potentials generated by brain activity in such experiments. Analyzing the resulting M/EEG recordings allows researchers to investigate how the brain processes information over time and across regions.


\progname{} aims to automate \href{https://mne.tools/stable/documentation/cookbook.html}{the typical M/EEG workflow}. Once the pipeline configuration is set up, user can simply call pipeline methods to execute analysis steps and access results without writing additional code. For example, user defines how to process raw data, events and epochs in the configuration, then it only takes a method call to get the resulting evoked data.




\subsection{Purpose of Document}


The purpose of this Software Requirements Specification (SRS) document is to define and describe the functional and non-functional requirements of the Eelbrain Pipeline project. This document serves as a reference for developers, to ensure a shared understanding of the software requirements. It also provides background information and planning for design, implementation, and validation.




\subsection{Scope of Requirements} 


The scope of \progname{} includes M/EEG data analyis (with or without MRI data) on a BIDS-compliant dataset. Its functionalities cover the typical M/EEG workflow. 




\subsection{Characteristics of Intended Reader} \label{sec_IntendedReader}


This document is for people interested in the development of the \progname{}. Readers should have a undergraduate level of knowledge in:


\begin{itemize}
    \item Neuroimaging data, particularly MEG/EEG data and its typical workflow
    \item The \href{https://bids-specification.readthedocs.io/en/stable/}{BIDS Specification}
    \item Software Engineering
\end{itemize}




\subsection{Organization of Document}


This document follows a abstract-to-specific approach. It first presents high-level goals, models and scope in the real world, then derivte the functional and nonfunctional requirements from them. For readers that are already familiar with the M/EEG domain, they can start with Section~\ref{Requirements} and look for clarifications in previous sections. This document follows the template for an SRS for scientific computing software proposed by \citet{SmithAndLai2005, SmithEtAl2007, SmithAndKoothoor2016}. 








\section{General System Description}


This section provides general information about the system.  It identifies the
interfaces between the system and its environment, describes the user
characteristics and lists the system constraints.




\subsection{System Context}


Ths system context is shown as inputs and outputs in Figure~\ref{Fig_SystemContext}. Eelbrain pipeline heavlily relies on MNE-Python as an external library for data processing.


\begin{figure}[h!]
\begin{center}
\includegraphics[width=0.6\textwidth]{SystemContextFigure}
\caption{System Context}
\label{Fig_SystemContext} 
\end{center}
\end{figure}


The responsibilities of each entity for data exchange are listed as follows.


\begin{itemize}
    \item User Responsibility: Ensure input dataset conforms to BIDS format.
    \item \progname{} Responsibilities:
        \begin{itemize}
            \item Detect data files in input dataset according to BIDS specification and provide correct file paths to MNE-Python functions.
            \item Transform experiment configuration defined by user into appropriate parameters for MNE-Python functions. Infer certain parameters when applicable.
        \end{itemize}
    \item MNE-Python Responsibility: Given valid file paths and parameters, apply data processing algorithms to M/EEG data correctly and efficiently.
\end{itemize}


The software will typically be used for neuroscience research. It will not be used for mission-critical or safety-critical work.




\subsection{User Characteristics} \label{SecUserCharacteristics}


The end user of \progname{} should have experience in:
\begin{itemize}
    \item Neuroimaging data, particularly MEG/EEG data and its typical workflow
    \item Python programming
\end{itemize}




\subsection{System Constraints}


\progname{} should delegate M/EEG data structures and data processing tasks to MNE-Python to leverage its well-tested and optimized implementations. 








\section{Specific System Description}


This section first presents the problem description, which gives a high-level view of the problem to be solved.  This is followed by the solution characteristics specification, which presents the assumptions, theories, definitions and finally the instance models.




\subsection{Problem Description} \label{Sec_pd}


\progname{} is intended to help researchers analysis M/EEG data more efficiently by automating the typical M/EEG workflow.




\subsubsection{Terminology and Definitions}


This subsection provides a list of terms that are used in the subsequent
sections and their meaning, with the purpose of reducing ambiguity and making it
easier to correctly understand the requirements:


\begin{itemize}

    \item \textbf{Neural source}: A region or population of neurons whose electrical activity contributes to the measured M/EEG signals.
    
    \item \textbf{Sensor}: A measurement device, such as an EEG electrode or MEG detector, used to record electromagnetic signals produced by neural activity.
    
    \item \textbf{Source space}: A predefined set of candidate locations within the brain where neural sources are allowed to exist for modeling purposes.

    \item \textbf{Forward model}: A physical and mathematical description of how neural source activity gives rise to signals measured by M/EEG sensors.
    
    \item \textbf{Inverse problem}: The problem of estimating neural source activity from M/EEG sensor measurements.
        
    \item \textbf{Source localization}: The process of estimating the locations of neural sources in the brain that generate the recorded M/EEG signals. Solving the inverse problem is a key step in source localization.

\end{itemize}




\subsubsection{Physical System Description} \label{sec_phySystDescrip}


The physical system of \progname{}, as shown in Figure~\ref{fig:physical_system}, includes the following elements


\begin{itemize}

    \item[PS1:] \textbf{Neural current sources.}  
    Electrical currents generated by neuronal populations within the cerebral cortex. These currents are the primary sources of the electromagnetic fields measured in M/EEG.
    % and are typically approximated as current dipoles with time-varying amplitudes.

    \item[PS2:] \textbf{Head volume conductor.}  
    The human head, consisting of brain, skull, and scalp tissues, which acts as a volume conductor for electromagnetic fields. The geometry and conductivity properties of these tissues influence how fields generated by neural sources propagate toward the sensors.

    \item[PS4:] \textbf{Sensor system.}  
    EEG electrodes and/or MEG sensors positioned on or around the head that measure electric potentials or magnetic fields resulting from neural activity.

    \item[PS5:] \textbf{External and instrumental noise.}  
    Environmental electromagnetic interference and sensor-related noise that contaminate the measured signals.
    % These effects are modeled statistically and influence the accuracy of source estimation.

\end{itemize}


\begin{figure}[h!]
\begin{center}
\includegraphics[width=0.7\textwidth]{physical_system.jpg}
\caption{Physical System of M/EEG}
\label{fig:physical_system}
\end{center}
\end{figure}




\subsubsection{Goal Statements}


\noindent Given the input M/EEG recordings and optional MRI data, the goal statements are:


\begin{itemize}

    \item[GS\refstepcounter{goalnum}\thegoalnum \label{G_sensorSpace}:] 
    Transform raw M/EEG recordings into epoched or evoked data to identify reliable temporal patterns of neural activity.

    \item[GS\refstepcounter{goalnum}\thegoalnum \label{G_sourceLocalization}:] Estimate source localization of neural activity from M/EEG recordings.

    \item[GS\refstepcounter{goalnum}\thegoalnum \label{G_noiseRobustness}:]
    Account for measurement and environmental noise when processing M/EEG data, when specified by user.

\end{itemize}




\subsection{Solution Characteristics Specification}


The instance models that govern \progname{} are presented in Subsection~\ref{sec_instance}.  The information to understand the meaning of the instance models and their derivation is also presented, so that the instance models can be verified.




\subsubsection{Types}




\subsubsection{Scope Decisions}




\subsubsection{Modelling Decisions}




\subsubsection{Assumptions} \label{sec_assumpt}


\begin{itemize}

    \item[A\refstepcounter{assumpnum}\theassumpnum \label{A_dipole}:]
    Neural electrical activity can be approximated by equivalent current dipoles distributed within the brain. [IM]

    \item[A\refstepcounter{assumpnum}\theassumpnum \label{A_linear}:]
    The relationship between neural source activity and M/EEG sensor measurements is linear. [IM]

    \item[A\refstepcounter{assumpnum}\theassumpnum \label{A_noiseGaussian}:]
    Measurement noise is assumed to be additive and can be modeled statistically using second-order moments. [IM]

\end{itemize}




\subsubsection{Theoretical Models}\label{sec_theoretical}


This section focuses on the general equations and laws that \progname{} is based on.


~\newline
\noindent
\label{TM:Optimization}
\deftheory
{ TM:Optimization }
{ Regularized linear inverse problem }
{
    $\hat{\mathbf{x}} =
    \arg\min_{\mathbf{x}}
    \left(
        \|\mathbf{y} - \mathbf{A}\mathbf{x}\|_2^2
        + \lambda\,\mathcal{R}(\mathbf{x})
    \right)$
}
{
    This theoretical model formulates a regularized linear inverse problem as an optimization task. The objective is to estimate an unknown vector $\hat{\mathbf{x}}$ that explains the observed data $\mathbf{y}$ through a linear operator $\mathbf{A}$. The first term enforces fidelity to the observed data, while the regularization term $\mathcal{R}(\mathbf{x})$ introduces additional constraints that ensure the existence, uniqueness, or stability of the solution. The regularization parameter $\lambda$ controls the relative importance of the regularization term. This formulation provides a general framework for solving ill-posed linear inverse problems.
}
{ None }
{ https://epubs.siam.org/doi/book/10.1137/1.9780898719697 }
{ \dref{IM_inverse} }
{ The linear operator $\mathbf{A}$ and observed data $\mathbf{y}$ are defined. }
~\newline




\subsubsection{General Definitions}\label{sec_gendef}




\subsubsection{Data Definitions}\label{sec_datadef}


This section collects and defines all the file types needed to build the instance models.


~\newline


\noindent
\begin{minipage}{\textwidth}
\renewcommand*{\arraystretch}{1.5}
\begin{tabular}{| p{\colAwidth} | p{\colBwidth}|}
\hline
\rowcolor[gray]{0.9}
    Number& DD\refstepcounter{datadefnum}\thedatadefnum \label{DD_numSensors} \\
\hline
    Label & \bf Number of sensors \\
\hline
    Symbol &$N_c$ \\
\hline
    SI Units & None \\
\hline
    Equation & None \\
\hline
    Description &
    $N_c$ is the total number of M/EEG sensors used to record the measurements. It determines the dimensionality of the sensor-level data vectors. \\
\hline
    Sources & None \\
\hline
    Ref.\ By & Section~\ref{sec_DataConstraints} \\
\hline
\end{tabular}
\end{minipage}\\
~\newline


\noindent
\begin{minipage}{\textwidth}
\renewcommand*{\arraystretch}{1.5}
\begin{tabular}{| p{\colAwidth} | p{\colBwidth}|}
\hline
\rowcolor[gray]{0.9}
    Number& DD\refstepcounter{datadefnum}\thedatadefnum \label{DD_samplingFreq}\\
\hline
    Label& \bf Sampling frequency\\
\hline
    Symbol &$f_s$\\
\hline
    SI Units & \si{\hertz}\\
\hline
    Equation& None \\
\hline
    Description &
    $f_s$ is the sampling frequency at which the M/EEG sensor signals are recorded. It determines the temporal resolution of the recorded data. \\
\hline
    Sources& None \\
\hline
    Ref.\ By & Section~\ref{sec_DataConstraints} \\
\hline
\end{tabular}
\end{minipage}\\
~\newline


\noindent
\begin{minipage}{\textwidth}
\renewcommand*{\arraystretch}{1.5}
\begin{tabular}{| p{\colAwidth} | p{\colBwidth}|}
\hline
\rowcolor[gray]{0.9}
    Number & DD\refstepcounter{datadefnum}\thedatadefnum \label{DD_rawSensorData}\\
\hline
    Label & \bf Raw sensor measurements\\
\hline
    Symbol & $\mathbf{y}(t)$\\
\hline
    SI Units & \si{\micro\volt} (EEG)\\
\hline
    Equation & None \\
\hline
    Description &
    $\mathbf{y}(t)$ is the vector of time-varying signals recorded at all sensors at time $t$, with each element corresponding to a single sensor measurement. \\
\hline
    Sources& None \\
\hline
    Ref.\ By & \iref{IM_sensorAggregate}, \iref{IM_inverseOpConstruct}, \iref{IM_inverse}, Section~\ref{sec_DataConstraints} \\
\hline
\end{tabular}
\end{minipage}\\
~\newline



\subsubsection{Data Types}\label{sec_datatypes}




\subsubsection{Instance Models} \label{sec_instance}    


This section transforms the problem defined in Section~\ref{Sec_pd} into  one which is expressed in mathematical terms. It uses concrete symbols defined  in Section~\ref{sec_datadef} to replace the abstract symbols in the models  identified in Sections~\ref{sec_theoretical} and~\ref{sec_gendef}.


The goal GS~\ref{G_sensorSpace} is solved by IM~\ref{IM_sensorAggregate}. The goals GS~\ref{G_sourceLocalization} and GS~\ref{G_noiseRobustness} are solved by IM~\ref{IM_inverse}.


~\newline
\noindent
\begin{minipage}{\textwidth}
\renewcommand{\arraystretch}{1.5}
\begin{tabular}{| p{\colAwidth} | p{\colBwidth}|}
\hline
\rowcolor[gray]{0.9}
    Number & IM\refstepcounter{instnum}\theinstnum \label{IM_sensorAggregate}\\
\hline
    Label& \bf Sensor space aggregation from raw data\\
\hline
    Input
    & Raw sensor measurements $\mathbf{y}(t)$; \\
    & Event information \\
\hline
    Output&
    Aggregated sensor-level representation $\mathbf{y}_a(t)$ \\
\hline
    Description
    & $\mathbf{y}(t)$ represents raw time-series measurements recorded at the sensors.\\
    & The data are transformed into an aggregated sensor-level representation $\mathbf{y}_a(t)$ through aggregation operations defined in the sensor space,
    such as averaging or other summary statistics over time or trials.\\
    & The resulting representation preserves the original sensor geometry while emphasizing consistent patterns across measurements.\\
    & This instance model specifies sensor space aggregation without introducing assumptions about the underlying source distribution.\\
\hline
    Sources & https://mitpress.mit.edu/9780262525855/an-introduction-to-the-event-related-potential-technique/ \\
\hline
    Ref.\ By & None \\
\hline
\end{tabular}
\end{minipage}\\
~\newline


~\newline
\noindent
\begin{minipage}{\textwidth}
\renewcommand{\arraystretch}{1.5}
\begin{tabular}{| p{\colAwidth} | p{\colBwidth}|}
\hline
\rowcolor[gray]{0.9}
    Number& IM\refstepcounter{instnum}\theinstnum \label{IM_inverseOpConstruct}\\
\hline
    Label& \bf Inverse operator construction \\
\hline
    Input
    & Raw MEG data $\mathbf{y}(t)$; \\
    & Structural MRI data; \\
    & Regularization parameter $\lambda$ \\
\hline
    Output
    & Inverse operator $\mathbf{W}$ such that \\
    & $\mathbf{W} =
        (\tilde{\mathbf{A}}^\top \tilde{\mathbf{A}} + \lambda \mathbf{I})^{-1}
        \tilde{\mathbf{A}}^\top$ \\
\hline
    Description &
    Raw MEG data $\mathbf{y}(t)$ are used to estimate the noise covariance matrix, which defines a whitening operator applied to the data and the forward model. \\
    & Structural MRI data are used to define the source space and to construct a geometrical head model, from which the linear forward operator $\mathbf{A}$ is obtained. \\
    & The whitened forward operator $\tilde{\mathbf{A}}$ is combined with the regularization parameter $\lambda$ to define the inverse operator $\mathbf{W}$ as the closed-form solution of a regularized linear inverse problem. \\
    & This instance model specifies how raw data and modeling assumptions determine a fixed inverse operator, independent of its subsequent application to sensor-level data. \\
\hline
    Sources & https://ui.adsabs.harvard.edu/abs/1993RvMP...65..413H/abstract \\
\hline
    Ref.\ By & \iref{IM_inverse} \\
\hline
\end{tabular}
\end{minipage}\\
~\newline


\noindent
\begin{minipage}{\textwidth}
\renewcommand{\arraystretch}{1.5}
\begin{tabular}{| p{\colAwidth} | p{\colBwidth}|}
\hline
\rowcolor[gray]{0.9}
    Number& IM\refstepcounter{instnum}\theinstnum \label{IM_inverse}\\
\hline
    Label& \bf Linear source estimation model\\
\hline
    Input
    & Observed M/EEG sensor measurements $\mathbf{y}(t)$. \\
    & Inverse operator $\mathbf{W}$. \\
    & Noise and regularization assumptions. \\
\hline
    Output
    & Estimated neural source activity $\hat{\mathbf{x}}(t)$ such that \\
    & $\hat{\mathbf{x}}(t) = \mathbf{W}\,\mathbf{y}(t)$ \\
\hline
    Description
    & $\mathbf{y}(t)$ represents the recorded sensor-level signals. \\
    & $\hat{\mathbf{x}}(t)$ represents the estimated neural source activity defined on the source space. \\
    & $\mathbf{W}$ is an inverse operator derived from the forward model and noise
    assumptions. \\
\hline
    Sources & https://pmc.ncbi.nlm.nih.gov/articles/PMC6870961/ \\
\hline
    Ref.\ By & None \\
\hline
\end{tabular}
\end{minipage}\\
~\newline


% \subsubsection*{Derivation of ...}




\subsubsection{Input Data Constraints} \label{sec_DataConstraints}    


Table~\ref{TblInputVar} shows the data constraints on the input output
variables.  The column for physical constraints gives the physical limitations
on the range of values that can be taken by the variable.  The column for
software constraints restricts the range of inputs to reasonable values.  The
software constraints will be helpful in the design stage for picking suitable
algorithms.  The constraints are conservative, to give the user of the model the
flexibility to experiment with unusual situations.  The column of typical values
is intended to provide a feel for a common scenario.  The uncertainty column
provides an estimate of the confidence with which the physical quantities can be
measured.  This information would be part of the input if one were performing an
uncertainty quantification exercise.


\begin{table}[!h]
\caption{Input Variables} \label{TblInputVar}
\renewcommand{\arraystretch}{1.2}
\noindent \begin{longtable*}{l l l l c} 
\toprule
\textbf{Var} & \textbf{Physical Constraints} & \textbf{Software Constraints} &
                            \textbf{Typical Value} & \textbf{Uncertainty}\\
\midrule 

    $f_s$ &
    $f_s > 0$ &
    $100 \leq f_s \leq 5000$ Hz &
    1000 Hz &
    0\% \\

    $N_c$ &
    $N_c \in \mathbb{N}$ &
    $1 \leq N_c \leq 400$ &
    64 &
    0\% \\

    ${\bf y}(t)$ &
    finite amplitude &
    $|{\bf y}(t)| \leq 500~\si{\micro\volt}$ (EEG) &
    50~\si{\micro\volt} &
    10\% \\

\bottomrule
\end{longtable*}
\end{table}




\subsubsection{Properties of a Correct Solution} \label{sec_CorrectSolution}

\noindent
A correct solution must exhibit:


\begin{itemize}

    \item \textbf{Linearity:}
    The solution must preserve the linear relationship between source activity and sensor measurements as defined in the instance models.

    \item \textbf{Spatial plausibility:}
    Estimated neural sources must lie within the predefined source space and not outside anatomically plausible regions.

\end{itemize}



\begin{table}[!h]
\caption{Output Variables} \label{TblOutputVar}
\renewcommand{\arraystretch}{1.2}
\noindent \begin{longtable*}{l p{0.65\textwidth}} 
\toprule
\textbf{Var} & \textbf{Physical Constraints} \\
\midrule

    $\hat{\bf x}(t)$ &
    Finite amplitude and defined only on the predefined source space \\

    ${\bf y}_{\text{model}}(t)$ &
    ${\bf y}_{\text{model}}(t) = {\bf L}\hat{\bf x}(t)$, preserving the linear 
    relationship \\

\bottomrule
\end{longtable*}
\end{table}








\section{Requirements} \label{Requirements}


This section provides the functional requirements, the business tasks that the software is expected to complete, and the nonfunctional requirements, the qualities that the software is expected to exhibit.




\subsection{Functional Requirements}


\noindent \begin{itemize}

    \item[R\refstepcounter{reqnum}\thereqnum \label{R_BIDS}:] The input dataset should conform to BIDS format. If MRI data is provided, it be in FreeSurfer format.
    
    \item[R\refstepcounter{reqnum}\thereqnum \label{R_Infer}:] Infer certain parameters from dataset when applicable.

    \item[R\refstepcounter{reqnum}\thereqnum \label{R_Configuration}:] Support sensor space analysis and source localization.

    \item[R\refstepcounter{reqnum}\thereqnum \label{R_Output}:] The output data should be in MNE-Python data structures representing different stages of processed data.

\end{itemize}




\subsection{Nonfunctional Requirements}


\noindent
\begin{itemize}

    \item[NFR\refstepcounter{nfrnum}\thenfrnum \label{NFR_Usability}:] \textbf{Usability}
    At least 80\% of users when surveyed will rate the software as easy to use.

    \item[NFR\refstepcounter{nfrnum}\thenfrnum \label{NFR_Maintainability}:]
    \textbf{Maintainability} For any developer with relavant domain knowledge, the effort required to make any of the likely changes listed for \progname{} should be less than 0.5 of the original development time.

    \item[NFR\refstepcounter{nfrnum}\thenfrnum \label{NFR_Portability}:]
    \textbf{Portability} \progname{} should run on any operating system (Windows, MacOS or Linux) with Python 3.8+ installed.


\end{itemize}




\subsection{Rationale}



The assumptions made are standard approaches adopted in the source papers of instance models, without which the brain response process can't be modeled mathematically.








\section{Likely Changes}    

\noindent \begin{itemize}

    \item[LC\refstepcounter{lcnum}\thelcnum\label{LC_bids}:] Support more BIDS entities in input dataset.

    \item[LC\refstepcounter{lcnum}\thelcnum\label{LC_preprocessing}:] Support more preprocessing operations.

    \item[LC\refstepcounter{lcnum}\thelcnum\label{LC_mri}:] Support more MRI formats other than FreeSurfer.

\end{itemize}








\section{Unlikely Changes}    

\noindent \begin{itemize}

    \item[LC\refstepcounter{lcnum}\thelcnum\label{LC_configuration}:] The format of input experiment configuration will not change.

    \item[LC\refstepcounter{lcnum}\thelcnum\label{LC_dataset}:] The input dataset format for raw data will not change.

\end{itemize}


\newpage








\section{Traceability Matrices and Graphs}


The purpose of the traceability matrices is to provide easy references on what
has to be additionally modified if a certain component is changed.  Every time a
component is changed, the items in the column of that component that are marked
with an ``X'' may have to be modified as well.  Table~\ref{Table:trace} shows the
dependencies of theoretical models, general definitions, data definitions, and
instance models with each other. Table~\ref{Table:R_trace} shows the
dependencies of instance models, requirements, and data constraints on each
other. Table~\ref{Table:A_trace} shows the dependencies of theoretical models,
general definitions, data definitions, instance models, and likely changes on
the assumptions.


\afterpage{
\begin{landscape}
\begin{table}[h!]
\centering
\begin{tabular}{|c|c|c|c|}
\hline
	& \aref{A_dipole} & \aref{A_linear} & \aref{A_noiseGaussian} \\
\hline
    \tref{TM:Optimization}   & & & \\ \hline
    \ddref{DD_numSensors}    & & & \\ \hline
    \ddref{DD_samplingFreq}  & & & \\ \hline
    \ddref{DD_rawSensorData} & & & \\ \hline
    \iref{IM_sensorAggregate}& X& X& X\\ \hline
    \iref{IM_inverseOpConstruct}& X& X& X\\ \hline
    \iref{IM_inverse}        & X& X& X\\ \hline
    \lcref{LC_bids}          & & & \\ \hline
    \lcref{LC_preprocessing} & & & \\ \hline
    \lcref{LC_mri}           & & & \\
\hline
\end{tabular}
\caption{Traceability Matrix Showing the Connections Between Assumptions and Other Items}
\label{Table:A_trace}
\end{table}
\end{landscape}
}


\begin{table}[h!]
\centering
\begin{tabular}{|c|c|c|c|c|c|c|c|}
\hline        
	& \tref{TM:Optimization} & \ddref{DD_numSensors} & \ddref{DD_samplingFreq} & \ddref{DD_rawSensorData} & \iref{IM_sensorAggregate} & \iref{IM_inverseOpConstruct} & \iref{IM_inverse} \\ \hline
    \tref{TM:Optimization}      & & & & & & & \\ \hline
    \ddref{DD_numSensors}       & & & & & & & \\ \hline
    \ddref{DD_samplingFreq}     & & & & & & & \\ \hline
    \ddref{DD_rawSensorData}    & & & & & & & \\ \hline
    \iref{IM_sensorAggregate}   & & & & X& & & \\ \hline
    \iref{IM_inverseOpConstruct}& & & & X& & & \\ \hline
    \iref{IM_inverse}           & X& & & X& & X& \\
\hline
\end{tabular}
\caption{Traceability Matrix Showing the Connections Between Items of Different Sections}
\label{Table:trace}
\end{table}


\begin{table}[h!]
\centering
\begin{tabular}{|c|c|c|c|c|c|c|c|c|}
\hline
	& \iref{IM_sensorAggregate}& \iref{IM_inverseOpConstruct}& \iref{IM_inverse}& \ref{sec_DataConstraints}& \rref{R_BIDS}& \rref{R_Infer}& \rref{R_Configuration}& \rref{R_Output} \\
\hline
    \iref{IM_sensorAggregate}    & & & & X& & & & \\ \hline
    \iref{IM_inverseOpConstruct} & & & & X& & & & \\ \hline
    \iref{IM_inverse}            & & X& & X& & & & \\ \hline
    \rref{R_BIDS}                & X& X& X& X& & & & \\ \hline
    \rref{R_Infer}               & X& X& & & & & & \\ \hline
    \rref{R_Configuration}       & X& & X& & & & & \\ \hline
    \rref{R_Output}              & X& X& X& & & & & \\
\hline
\end{tabular}
\caption{Traceability Matrix Showing the Connections Between Requirements and Instance Models}
\label{Table:R_trace}
\end{table}


% The purpose of the traceability graphs is also to provide easy references on what has to be additionally modified if a certain component is changed.  The arrows in the graphs represent dependencies. The component at the tail of an arrow is depended on by the component at the head of that arrow. Therefore, if a component is changed, the components that it points to should also be changed. Figure~\ref{Fig_ATrace} shows the dependencies of theoretical models, general definitions, data definitions, instance models, likely changes, and assumptions on each other. Figure~\ref{Fig_RTrace} shows the dependencies of instance models, requirements, and data constraints on each other.


% \begin{figure}[h!]
% 	\begin{center}
% 		%\rotatebox{-90}
% 		{
% 			\includegraphics[width=\textwidth]{ATrace.png}
% 		}
% 		\caption{\label{Fig_ATrace} Traceability Matrix Showing the Connections Between Items of Different Sections}
% 	\end{center}
% \end{figure}


% \begin{figure}[h!]
% 	\begin{center}
% 		%\rotatebox{-90}
% 		{
% 			\includegraphics[width=0.7\textwidth]{RTrace.png}
% 		}
% 		\caption{\label{Fig_RTrace} Traceability Matrix Showing the Connections Between Requirements, Instance Models, and Data Constraints}
% 	\end{center}
% \end{figure}








\section{Development Plan}








\section{Values of Auxiliary Constants}


\newpage








\bibliographystyle {plainnat}
\bibliography {../../refs/References}


\end{document}